\chapter*{Introduction}
\label{ch:Introduction}
This report illustrates development and findings of Group 2's project for
the course Text Analytics, Academic Year 2024-2025.

\section*{Team members and roles}


% copy-pasted from project proposal
\section*{Motivation and project goal}
Songs have the unique ability to engage people in ways that few other
mediums can match. While beats and melodies contribute to this impact,
it is often the lyrics that give songs their true emotional strength.
Lyrics serve as one of the main foundations of songs, playing
a crucial role in expressing feelings in many different ways.\\

Despite this, the focus is often shifted more on song genres, themes,
or artists, rather than on the specific emotions conveyed through lyrics.
Analyzing the emotional tone of song texts can give useful insights about cultural trends, social issues,
individual mental states and more. Furthermore, this type of analysis
could help identify how specific
emotional responses are evoked, making it a useful tool across various fields.\\

The main goal of this project is to perform emotion
detection on stanzas of songs. Lyrics' emotional tone can be useful
for many things, such as generating potentially useful information for automatized
playlist creation and songs' organization.\\
The decision to perform the task on stanzas rather than songs was taken in order
to gain a deeper understanding of the fluctuations of emotions inside
songs, as songs often describe complicated situations and states of mind.\\




Furthermore, we intend to compare different textual preprocessing and
Machine Learning models, in order to explore different possibilities and
evaluate their performances.

\section*{Chapter overview}

\chapter*{Introduction}
\label{ch:Introduction}
This report illustrates development and findings of Group 2's project for
the Text Analytics course, Academic Year 2024/2025. \\

% \section*{Team members and roles}


% copy-pasted from project proposal
%\section*{Motivation and project goal}

% Songs have the unique ability to engage people in ways that few other
% mediums can match. While beats and melodies contribute to this impact,
% it is often the lyrics that give songs their true emotional strength.
Lyrics serve as one of the main foundations of songs, playing a crucial role in expressing feelings in many different ways. The emotional tone of songs can serve various purposes, such as automatized playlist creation or songs' organization,
offering an alternative to the more traditional genre-based classification. \\
%Analyzing the emotional tone of song texts can give useful insights about individual mental states, cultural trends, social issues and more.
% The main goal of this project is to perform emotion detection on stanzas of songs.
The goal of this project is the development of 4 Machine Learning models that perform emotion detection on song lyrics stanzas. To obtain a deeper understanding of emotional fluctuations within the texts, the models assign emotion labels to individual stanzas instead of full songs.
The emotion labels are assigned based on Robert Plutchik's eight primary emotions (shown in figure~\ref{fig:primary_emotions}), offering a comprehensive range for representing diverse emotional states.\\
\begin{figure}[H]
    \centering
    \includegraphics[scale= 0.30]{pictures/plutchik_primary_emotions.png}
    \caption{Plutchik's eight primary emotions}
    \label{fig:primary_emotions}
\end{figure}

This report tries to cover and illustrate clearly various aspects of this work. 
In the \textit{Method} section, we will provide a detailed explanation of the data and procedures used in the project, in particular describing the pipeline taken to implement the models.
Then, the \textit{Results} chapter will provide an overview of the obtained results with the aid of plots and figures, highlighting the significant outcomes.
This section will be connected to the last two, i.e. the \textit{Discussion} and \textit{Conclusions} ones, which will explain what the general findings mean, recapping the primary objective of the work and discussing the importance or potential applications of the results.
% Furthermore, we intend to compare different textual preprocessing and
% Machine Learning models, in order to explore different possibilities and
% evaluate their performances.


%\section*{Dataset overview}

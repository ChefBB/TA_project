\chapter*{Methods}
\label{ch:capitolo2}

% This section will provide an overview about the data, methods and procedures followed in the project.\\

The dataset used in this project is a sampled subset of English-language
songs derived from the \textit{Genius Song Lyrics Dataset}\textsuperscript{\cite{geniusdataset}}.
% The original dataset (3m records) included songs in many different languages;
% however, this work focused exclusively on English-language ones.
The original dataset contained numerous attributes; the ones considered
relevant for model training are:
\begin{itemize}
    \item \textbf{title:} the song's title;
    % \item \textbf{artist:} the artist
    % \item \textbf{year:}
    % \item \textbf{views:}
    % \item \textbf{features:}

    % \item \textbf{id:}
    \item \textbf{lemmatized\_stanzas:} lyrics of the single stanza;
    
    \item \textbf{stanza\_number:} identifies the position of the stanza in the song;

    \item \textbf{is\_chorus:} boolean variable that attests whether the stanza is
        a chorus or not;
    
    \item \textbf{is\_country, is\_pop, is\_rap, is\_rb, is\_rock:} boolean variables, result of a one-hot encoding process, that represent songs genres;

    \item \textbf{label:} represents the emotional classification of the stanza,
        assigned by Albert Base v2\textsuperscript{\cite{albert-base-v2}} model.
    
    % \item \textbf{is\_country:}
    % \item \textbf{is\_pop:}
    % \item \textbf{is\_rap:}
    % \item \textbf{is\_rb:}
    % \item \textbf{is\_rock:}
\end{itemize}

% The dataset originally didn't contain emotion labels, essential for training the models.
% To create the ground truth, the model
% Albert Base v2\textsuperscript{\cite{albert-base-v2}} was used, classifying
% stanzas' lyrics into Plutchik's eight primary emotions.
All of these attributes, except for \texttt{title}, were the result
of the preprocessing phase, as described in section~\ref{preprocessing}.
Due to limited computational power, the labeling process was time-intensive,
ultimately resulting in a limited dataset, with a few more than 100.000 entries.

%PREPROCESSING
\section*{Preprocessing}
\label{preprocessing}
The initial preprocessing step involved sampling from the original dataset
while maintaining the proportional distribution of genres.
This approach ensured that the genre representation in the sampled subset
accurately reflected that of the full dataset.\\

The preliminary text cleaning process focused on the \texttt{lyrics} attribute,
which contained the complete lyrics of each song in string format.
Initially, a regular expression (RegEx) was built to remove noise from the
lyrics, specifically targeting words enclosed in square brackets that were
irrelevant to the stanza splitting process. Many keywords marking different
stanzas were written within square brackets, and removing non-keyword items
inside brackets was crucial to avoid potential issues.\\

The next critical step was stanza splitting. After cleaning texts from
noisy square-bracketed items, lyrics were split based on various keywords
used to denote stanzas (such as "chorus", "verse", "intro", "outro", "refrain", "hook", etc.).
The RegEx developed accounted for the different formats in which these keywords
appeared, including square brackets, parentheses, or no brackets at all, as well
as stanzas separated only by double newline characters.
The output of this step was, for each song record, a list of strings
representing individual stanzas (each stanza has also a header with the corresponding
keyword; this aspect will be discussed in the next paragraph).
Next, uninformative strings—such as empty strings or those with fewer
than 20 characters—were removed, as they were too short to provide meaningful
content.
As a result, the output of this preliminary preprocessing phase was a dataset
where the records were no longer whole songs but individual stanzas, each
numbered according to its position within the song.\\

A further and more detailed cleaning process on the stanzas involved the creation
of the boolean feature \texttt{is\_chorus}, which was assigned a \texttt{true}
value for repeated stanzas within the same song or for stanzas with headers such
as "hook", "chorus", "refrain", or "bridge".
Next, stanza headers and newline characters between verses were removed to obtain
cleaner stanzas.
Since choruses, hooks, bridges, and refrains often repeat throughout songs,
duplicate stanzas were discarded to avoid redundant data. This resulted in a
dataset of cleaned, non-duplicate stanzas, which served as the checkpoint for
the labeling step and the starting point for the text lemmatization process.\\

The subsequent step involved lemmatizing the stanzas using the \texttt{spaCy}
library. A list of lemmatized tokens was created by filtering out punctuation
and empty words. Lemmatization was chosen over stemming because it produces
more accurate and meaningful results, particularly for tasks requiring semantic
understanding, such as the one at hand.\\

Since the dataset was not pre-labeled at the stanza level, ALBERT Base v2 was employed for this task.
This transformer model is specifically designed to be fine-tuned on tasks that
require an understanding of the entire sentence, such as sequence classification.

\section*{Models}
The selected model architectures are:
\begin{itemize} 
    \item \textbf{Random Forest}: A robust ensemble learning method known for
    its ability to handle complex, high-dimensional datasets effectively.
    \item \textbf{Support Vector Machine (SVM)}: A powerful classifier that
    excels in separating classes by finding the optimal hyperplane,
    particularly effective in text classification tasks.
    \item \textbf{One-Dimensional Convolutional Neural Network (1D-CNN)}:
    Designed to capture local patterns in sequential data, leveraging
    convolutional layers to learn hierarchical features.
    \item \textbf{Recurrent Neural Network (RNN)}: Utilized for its strength
    in processing sequential data, with the ability to capture contextual
    relationships between words across different stanzas.
\end{itemize}
Their different approaches and depths are an important point of the study, as they
offer interesting insights into the possible different techniques and levels of
complexity required for detecting emotional tones in complex pieces of text.

%MODELLI STATICI
\subsection*{Static Models}
The development of static models was then simple and straight forward.
Static models are here meant to provide a performance comparison
for the more complex Neural Networks.
The two architectures are the same, consisting of a preprocessing
layer to handle the inputs, followed by the classifier itself.\\

The preprocessing layers handle both \texttt{title} and
\texttt{lemmatized\_stanzas} through TF-IDF for feature extraction.
An additional analysis aimed at identifying the most significant features for
each emotion label in the dataset was conducted to enhance classification.
Feature importance analysis was performed on the already labeled and
lemmatized dataset, following these steps:
\begin{itemize}
    \item \textbf{Custom Stopword List Creation}: A custom stopword list was
    compiled, consisting
    of the most frequent and generic words in the dataset, along with additional
    punctuation marks and common typographical errors not covered by the default
    NLTK stopword lists.

    \item \textbf{Stopword Removal}: Irrelevant words identified by the custom
    stopword list were removed from the lemmatized stanzas to reduce noise in
    the data.

    \item \textbf{TF-IDF Analysis per Label}: A function was developed to
    compute TF-IDF scores for a given text, with parameters \texttt{min\_df}
    set to 2 and \texttt{max\_df} set to 0.80.
\end{itemize}
This configuration ensured that words appearing in fewer than two or
more than 80\% of the documents were ignored, minimizing the influence of
extremely rare or overly common words. The function was applied separately to
the cleaned stanzas for each label, with the aim of identifying the most
relevant features per emotion category.
The results, however, did not meet expectations, though the outcome was not
entirely surprising. Most labels shared at least two common features, and
certain labels (such as surprise and trust) shared all features.
Additionally, all identified features exhibited very low TF-IDF scores,
below 0.05.
This result appears to be inherent to the nature of the dataset: song lyrics
frequently contain repetitive and generic language, making it difficult to
distinguish specific emotions based solely on textual features.
Consequently, the analysis was concluded at this point.\\

Random Search was chosen for hyperparameter tuning, for both models.
Cross validation is also used in order to provide a more accurate
estimate of model performance.

\subsection*{Neural Networks}
The architectures were developed and tuned through empirical,
reiterated testing. These parameters helped with the process, and can
be used for further experimentation.
Both the Recurrent Neural Network and the One-Dimensional Convolutional
Neural Network share the same preprocessing
architecture. Most attributes are processed in the same manner as in
the Static Models; specific steps are applied to \texttt{lemmatized\_stanzas}
and \texttt{title}.
Non-Negative Matrix Factorization is applied to \texttt{title} in addition to
Term Frequency-Inverse Document Frequency
to extract latent topics, providing a richer representation of the
textual data.\\

\texttt{lemmatized\_stanzas} are handled by Convolutional and Recurrent
pipelines of the two Networks.
Elements are first tokenized, and then padded in order to
get an input with consistent shape, which is essential for both types of
recurrent layers.\\

\subsubsection*{One-Dimensional Convolutional Neural Network}
The Convolutional part of the architecture is specifically designed to extract and learn
local patterns in \texttt{embedding\_lyrics}.
Its structure consists of three convolutional layers, each applying filters of
varying sizes. This allows to detect patterns at different granularities.
These layers are followed by Global Max Pooling, to reduce the previous output's
dimension to a fixed-length vector, as well as retaining focus on the most
informative patterns.
A dropout layer is then applied, to introduce regularization and prevent
overfitting.


\subsubsection*{Recurrent Neural Network}
The Recurrent part of the architecture is specifically designed to extract and learn
local patterns in \texttt{embedding\_lyrics}.
Its structure consists of three Gated Recurrent Units (\texttt{GRU} layers)
to model temporal relationships. These are characterized by progressively smaller
numbers of units; this allows pattern capture at different abstraction levels.
All three layers in the architecture use the \texttt{tanh} activation function
to compute the hidden state and
the \texttt{sigmoid} activation function for the recurrent gate.
The first and second layers return the full sequence of hidden states for each
time step in the input sequence, enabling richer learning of patterns over time.
Dropout is applied on every layer, to prevent overfitting and add regularization.


\subsection*{Shared Components}
The remaining features are handled by a simple pipeline, which concatenates
their Input layers.
The inputs are passed them through a dense layer to create a compact
representation.
The output of the lyrics-processing branch is concatenated with the processed
additional features.
The combined representation is passed through a Dense layer with 32 units
(ReLU activation), followed by a Dropout layer with a rate of 0.3.
Finally, the output layer uses 8 units with a softmax activation, corresponding
to the classification into 8 emotion categories.\\


The models are trained using categorical cross-entropy as the loss function,
as it consistently produced better-performing results for both types of
neural networks.
As evaluation metric, the better performing one was categorical accuracy.
Other metrics were tested for evaluation purposes, particularly
top-k categorical accuracy with $k=2$, which yielded interesting insights
by considering a prediction correct if the true label is among the top two
predicted classes. While it showed potential in improving generalization and
preventing overfitting, it was ultimately discarded as a primary evaluation
metric. This is because, with $k=2$, the model has a 25\% baseline
chance of being correct when there are eight possible classes, which reduces
the precision required for accurate learning and limits performance improvements.
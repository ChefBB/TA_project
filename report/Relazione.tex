% Carattere dimensione 12
\documentclass[12pt]{report}

% Per la stampa fronte-retro sostituire con:
% \documentclass[12pt, twoside]{report}

% Margini (4cm a sx, 2.5cm a dx, 2.5cm in alto, 2.5cm in basso)
\usepackage[top=2.5cm, bottom=2.5cm, left=4cm, right=2.5cm, centering]{geometry}

% Per la stampa fronte-retro sostituire con: 
% \usepackage[top=2.5cm, bottom=2.5cm, inner=4cm, outer=4cm, right=2.5cm, centering]{geometry}

% Interlinea
\linespread{1.5}

% Librerie utili
\usepackage[italian]{babel} % applicazione regole di scrittura per la lingua italiana 
\usepackage[utf8]{inputenc} % codifica UTF-8
\usepackage{scrlayer-scrpage} % stili pagina per il frontespizio
\ifoot[]{}
\cfoot[]{}
\ofoot[\pagemark]{\pagemark}
\pagestyle{scrplain}
\usepackage{mathptmx} % font Times New Roman (simile)
\usepackage{graphicx} % inserimento di immagini
\usepackage{csquotes} % per le citazioni "in blocco"
\usepackage[backend=biber, sorting=none, ]{biblatex} % bibliografia con pacchetto biblatex (https://ctan.org/pkg/biblatex?lang=en)
\addbibresource{bibliography.bib}
\appto{\bibsetup}{\raggedright}

\usepackage{titlesec} % per la formattazione dei titoli delle sezioni, capitoli etc.
\usepackage{float} % per il posizionamento delle immagini

\usepackage{listings} % per il codice di programmazione
% Fonte https://en.wikibooks.org/wiki/LaTeX/Source_Code_Listings. Per la lista di sintassi riconosciute.
\renewcommand{\lstlistingname}{Code}% Listing -> Codice
\usepackage{xcolor}  % stile del codice
\definecolor{mygreen}{rgb}{0,0.6,0}
\definecolor{mygray}{rgb}{0.5,0.5,0.5}
\definecolor{mymauve}{rgb}{0.58,0,0.82}
\definecolor{darkgray}{rgb}{.4,.4,.4}
\definecolor{navy}{HTML}{000080}
\definecolor{purple}{rgb}{0.65, 0.12, 0.82}
\definecolor{codepurple}{rgb}{0.58,0,0.82}
\definecolor{backcolour}{rgb}{0.95,0.95,0.92}

% \usepackage{longtable}
\usepackage{tabularx}


% Stili configurabili del codice (lslisting) 
\lstset{ %
belowcaptionskip=0.5em,
backgroundcolor=\color{backcolour}, % choose the background color; you must add \usepackage{color} or \usepackage{xcolor}
basicstyle=\footnotesize, % the size of the fonts that are used for the code
breakatwhitespace=false, % sets if automatic breaks should only happen at whitespace
breaklines=true, % sets automatic line breaking
captionpos=b, % sets the caption-position to bottom
commentstyle=\color{mygreen}, % comment style
deletekeywords={...}, % if you want to delete keywords from the given language
escapeinside={\%*}{*)}, % if you want to add LaTeX within your code
extendedchars=true, % lets you use non-ASCII characters; for 8-bits encodings only, does not work with UTF-8
frame=single, % adds a frame around the code
keepspaces=true, % keeps spaces in text, useful for keeping indentation of code (possibly needs columns=flexible)
keywordstyle=\color{codepurple}, % keyword style
% language=Octave, % the language of the code
morekeywords={*,...}, % if you want to add more keywords to the set
numbers=left, % where to put the line-numbers; possible values are (none, left, right)
numbersep=5pt, % how far the line-numbers are from the code
numberstyle=\tiny\color{mygray}, % the style that is used for the line-numbers
rulecolor=\color{black}, % if not set, the frame-color may be changed on line-breaks within not-black text (e.g. comments (green here))
showspaces=false, % show spaces everywhere adding particular underscores; it overrides 'showstringspaces'
showstringspaces=false, % underline spaces within strings only
showtabs=false, % show tabs within strings adding particular underscores
stepnumber=1, % the step between two line-numbers. If it's 1, each line will be numbered
stringstyle=\color{mymauve}, % string literal style
tabsize=2, % sets default tabsize to 2 spaces
title=\lstname % show the filename of files included with \lstinputlisting; also try caption instead of title
}


\setlength{\parindent}{0pt}


% END of listing package 

% Formato delle intestazioni
\titleformat{\chapter}[block]
  {\normalfont\LARGE\bfseries}{\thechapter.}{0.5em}{\LARGE}
\titlespacing*{\chapter}{0pt}{-20pt}{25pt}

\begin{document}

% Frontespizio
\begin{titlepage}
% \begin{figure}
%     \centering\includegraphics[scale=0.5]{immagini/cherubino_pant541.png}
% \end{figure}

\begin{center}
    {\LARGE{ Emotion Detection on song lyrics stanzas \\}}
    \vspace{3cm}
    {\Large { TEXT ANALYTICS }}\\
    \vspace{1.5cm}
    {\Large { Group 2 }}\\
    \vspace{1.5cm}
    {\Large { Barbieri, Bosco, Ferrara, Rotellini, Zizza }}
\end{center}

% \centering{\large{\bf ANNO ACCADEMICO 20xx/20xx }}
\end{titlepage}
% Fine frontespizio

\renewcommand{\contentsname}{Index}
\tableofcontents
\thispagestyle{empty}


\renewcommand{\listfigurename}{List of figures}
\listoffigures

\thispagestyle{empty}
\clearpage
\setcounter{page}{1}
\addtocontents{toc}{\protect\thispagestyle{empty}}
\addcontentsline{toc}{chapter}{Introduction} % Capitolo non numerato
\chapter*{Introduction}
\label{ch:Introduction}
This report illustrates development and findings of Group 2's project for
the Text Analytics course, Academic Year 2024/2025. \\

% \section*{Team members and roles}


% copy-pasted from project proposal
%\section*{Motivation and project goal}

% Songs have the unique ability to engage people in ways that few other
% mediums can match. While beats and melodies contribute to this impact,
% it is often the lyrics that give songs their true emotional strength.
Lyrics serve as one of the main foundations of songs, playing a crucial role in expressing feelings in many different ways. The emotional tone of songs can serve various purposes, such as automatized playlist creation or songs' organization,
offering an alternative to the more traditional genre-based classification. \\
%Analyzing the emotional tone of song texts can give useful insights about individual mental states, cultural trends, social issues and more.
% The main goal of this project is to perform emotion detection on stanzas of songs.
The goal of this project is the development of 4 Machine Learning models that perform emotion detection on song lyrics stanzas. To obtain a deeper understanding of emotional fluctuations within the texts, the models assign emotion labels to individual stanzas instead of full songs.
The emotion labels are assigned based on Robert Plutchik's eight primary emotions (shown in figure~\ref{fig:primary_emotions}), offering a comprehensive range for representing diverse emotional states.\\
\begin{figure}[H]
    \centering
    \includegraphics[scale= 0.30]{pictures/plutchik_primary_emotions.png}
    \caption{Plutchik's eight primary emotions}
    \label{fig:primary_emotions}
\end{figure}

This report tries to cover and illustrate clearly various aspects of this work. 
In the \textit{Method} section, we will provide a detailed explanation of the data and procedures used in the project, in particular describing the pipeline taken to implement the models.
Then, the \textit{Results} chapter will provide an overview of the obtained results with the aid of plots and figures, highlighting the significant outcomes.
This section will be connected to the last two, i.e. the \textit{Discussion} and \textit{Conclusions} ones, which will explain what the general findings mean, recapping the primary objective of the work and discussing the importance or potential applications of the results.
% Furthermore, we intend to compare different textual preprocessing and
% Machine Learning models, in order to explore different possibilities and
% evaluate their performances.


%\section*{Dataset overview}


\clearpage

% Dataset overview
% !!! cut, content in intro
% % !!!
% cut at the moment; content moved to introduction

\chapter{Dataset overview}
\label{ch:capitolo1}

% copy-pasted from project proposal
The dataset utilized in this project represents a subset of songs
derived from the Genius Song Lyrics Dataset\textsuperscript{\cite{geniusdataset}}.
The dataset contains 11 attributes
that represent various song data, including the lyrics.
The original dataset includes songs in all languages: for our aim
we will be using the english ones only.\\

The dataset doesn't have emotion labels, which are essential for training the models.
To create the ground truth, the model
Albert Base v2\textsuperscript{\cite{albert-base-v2}} was used, classifying
stanzas' lyrics with Plutchik's eight primary emotions.

% The stanzas are labeled using Robert Plutchik's 8 primary emotions;
% the emotions included in this representation are:
% anger, fear, sadness, disgust, surprise, anticipation, trust, and joy.
% Such multifaceted emotions allow us to finely analyze the feelings and
% moods conveyed by songs.
% 
% \clearpage

% Preprocessing
\chapter{Preprocessing}
\label{ch:capitolo2}

The first step in the preprocessing phase involved sampling the original dataset while preserving the original proportions of the different genres. 
This ensured that the genre distribution in the subset remained representative of the full dataset.


\clearpage

% static models
\chapter{Static Models}
\label{ch:static_models}


\section{Random Forest}

\section{SVM}


\clearpage

% neural networks
\chapter{Neural Networks}
\label{ch:capitolo4}

\section{One-Dimensional Convolutional Neural Network}

\section{Recurrent Neural Network}

\clearpage

\addcontentsline{toc}{chapter}{Key findings and conclusions} % Capitolo non numerato
\input{conclusioni.tex}
\clearpage

% \renewcommand{\contentsname}{Bibliography}
% \bibliographystyle{plain} % We choose the "plain" reference style
% \bibliography{bibliography} % Entries are in the refs.bib file
% \addbibresource{bibliography.bib}
\printbibliography

\end{document}
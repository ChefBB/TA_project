% Carattere dimensione 12
\documentclass[12pt,openany]{report}

% Per la stampa fronte-retro sostituire con:
% \documentclass[12pt, twoside]{report}

% Margini (4cm a sx, 2.5cm a dx, 2.5cm in alto, 2.5cm in basso)
\usepackage[top=2.5cm, bottom=2.5cm, left=2.5cm, right=2.5cm, centering]{geometry}

% Per la stampa fronte-retro sostituire con: 
% \usepackage[top=2.5cm, bottom=2.5cm, inner=4cm, outer=4cm, right=2.5cm, centering]{geometry}

% Interlinea
\linespread{1.5}

% Librerie utili
\usepackage[english]{babel} % applicazione regole di scrittura per la lingua italiana 
\usepackage[utf8]{inputenc} % codifica UTF-8
\usepackage{scrlayer-scrpage} % stili pagina per il frontespizio
\ifoot[]{}
\cfoot[]{}
\cfoot[\pagemark]{\pagemark}
\pagestyle{scrplain}
\usepackage{mathptmx} % font Times New Roman (simile)
\usepackage{graphicx} % inserimento di immagini
\usepackage{csquotes} % per le citazioni "in blocco"
\usepackage[backend=biber, sorting=none, ]{biblatex} % bibliografia con pacchetto biblatex (https://ctan.org/pkg/biblatex?lang=en)
\addbibresource{bibliography.bib}
\appto{\bibsetup}{\raggedright}

\usepackage{titlesec} % per la formattazione dei titoli delle sezioni, capitoli etc.
\usepackage{float} % per il posizionamento delle immagini

\usepackage{listings} % per il codice di programmazione
% Fonte https://en.wikibooks.org/wiki/LaTeX/Source_Code_Listings. Per la lista di sintassi riconosciute.
\renewcommand{\lstlistingname}{Code}% Listing -> Codice
\usepackage{xcolor}  % stile del codice
\definecolor{mygreen}{rgb}{0,0.6,0}
\definecolor{mygray}{rgb}{0.5,0.5,0.5}
\definecolor{mymauve}{rgb}{0.58,0,0.82}
\definecolor{darkgray}{rgb}{.4,.4,.4}
\definecolor{navy}{HTML}{000080}
\definecolor{purple}{rgb}{0.65, 0.12, 0.82}
\definecolor{codepurple}{rgb}{0.58,0,0.82}
\definecolor{backcolour}{rgb}{0.95,0.95,0.92}

% \usepackage{longtable}
\usepackage{tabularx}


% Stili configurabili del codice (lslisting) 
\lstset{ %
belowcaptionskip=0.5em,
backgroundcolor=\color{backcolour}, % choose the background color; you must add \usepackage{color} or \usepackage{xcolor}
basicstyle=\footnotesize, % the size of the fonts that are used for the code
breakatwhitespace=false, % sets if automatic breaks should only happen at whitespace
breaklines=true, % sets automatic line breaking
captionpos=b, % sets the caption-position to bottom
commentstyle=\color{mygreen}, % comment style
deletekeywords={...}, % if you want to delete keywords from the given language
escapeinside={\%*}{*)}, % if you want to add LaTeX within your code
extendedchars=true, % lets you use non-ASCII characters; for 8-bits encodings only, does not work with UTF-8
frame=single, % adds a frame around the code
keepspaces=true, % keeps spaces in text, useful for keeping indentation of code (possibly needs columns=flexible)
keywordstyle=\color{codepurple}, % keyword style
% language=Octave, % the language of the code
morekeywords={*,...}, % if you want to add more keywords to the set
numbers=left, % where to put the line-numbers; possible values are (none, left, right)
numbersep=5pt, % how far the line-numbers are from the code
numberstyle=\tiny\color{mygray}, % the style that is used for the line-numbers
rulecolor=\color{black}, % if not set, the frame-color may be changed on line-breaks within not-black text (e.g. comments (green here))
showspaces=false, % show spaces everywhere adding particular underscores; it overrides 'showstringspaces'
showstringspaces=false, % underline spaces within strings only
showtabs=false, % show tabs within strings adding particular underscores
stepnumber=1, % the step between two line-numbers. If it's 1, each line will be numbered
stringstyle=\color{mymauve}, % string literal style
tabsize=2, % sets default tabsize to 2 spaces
title=\lstname % show the filename of files included with \lstinputlisting; also try caption instead of title
}


\setlength{\parindent}{0pt}


% END of listing package

\makeatletter
\renewcommand{\chapter}{%
  \thispagestyle{plain}% Optional: Adjust page style
  \global\@topnum\z@
  \@afterindentfalse
  \secdef\@schapter\@schapter}
\makeatother

% Formato delle intestazioni
\titleformat{\chapter}[hang]
  {\normalfont\LARGE\bfseries}{\thechapter.}{0.5em}{\LARGE}
\titlespacing*{\chapter}{0pt}{30pt}{25pt}

\newcommand{\firstchapter}{
  \titlespacing*{\chapter}{0pt}{0pt}{25pt} % Reduce space before first chapter title
}


% \titleformat{\chapter}[hang]{\normalfont\huge\bfseries}{\thechapter}{1em}{}
% \titlespacing*{\chapter}{0pt}{-2em}{1em}


\begin{document}

% Frontespizio
% \begin{titlepage}
% \begin{figure}
%     \centering\includegraphics[scale=0.5]{immagini/cherubino_pant541.png}
% \end{figure}

% \begin{center}
%     {\LARGE{ Emotion Detection in Song Lyrics Stanzas \\}}
%     \vspace{3cm}
%     {\Large { TEXT ANALYTICS - A.Y. 2024/2025 }}\\
%     \vspace{1.5cm}
%     {\Large { Group 2 }}\\
%     \vspace{1.5cm}
%     {\Large { Barbieri, Bosco, Ferrara, Rotellini, Zizza }}
% \end{center}

% \centering{\large{\bf ANNO ACCADEMICO 20xx/20xx }}
% \end{titlepage}
% Fine frontespizio


\setcounter{page}{1}
\addtocontents{toc}{\protect\thispagestyle{empty}}
\addcontentsline{toc}{chapter}{Introduction} % Capitolo non numerato

% Abstract section
\chapter*{Abstract}
% Analyzing the emotional tone of songs texts can give insights into
% societal trends and this information can be useful especially for recommendation
% algorithms.
% This study aims to build four Machine Learning models that classify
% emotions expressed in English song lyrics at the stanza level;
% the emotion labels are given according to
% Plutchik's eight primary emotions.\\

% The chosen module architectures are the following:
% \begin{itemize}
%     \item \textbf{Random Forest}
%     \item \textbf{SVM}
%     \item \textbf{One-Dimensional Convolutional Neural Network}
%     \item \textbf{Recurrent Neural Network}
% \end{itemize}

% These were chosen for their popularity and wide applicability, as well as
% using different conceptual approaches to solve the problem.
Analyzing the emotional tone of songs' texts can give insights into societal trends and this information can be useful especially for recommendation
algorithms. This study focuses on developing four Machine Learning models to classify
emotions conveyed in English song lyrics at the stanza level. The
classification employs Plutchik's eight primary emotions,
offering a nuanced understanding of emotional expression in lyrical content.
The selected model architectures are:
\begin{itemize}
    \item \textbf{Random Forest}
    \item \textbf{Support Vector Machine (SVM)}
    \item \textbf{One-Dimensional Convolutional Neural Network (1D-CNN)}
    \item \textbf{Recurrent Neural Network (RNN)}
\end{itemize}
%MI SEMBRAVANO AGGIUNTE RIDONDANTI
%These models were chosen for their proven effectiveness across various
%domains and their diverse approaches, providing a thorough
%investigation of different techniques for emotion classification in text.\\

% \section*{Abstract}
% Analyzing the emotional tone of songs texts can give insights into
% societal trends and this information can be useful especially for recommendation
% algorithms.
% This study aims to build 4 Machine Learning models that classify
% emotions expressed in English song lyrics at the stanza level; 
% the emotion labels are given according to
% Plutchik's 8 primary emotions.
\textbf{DA FINIRE CON CONCLUSIONI}

% Introduction (on the same page)
\chapter*{Introduction}
\label{ch:Introduction}
This report illustrates development and findings of Group 2's project for
the Text Analytics course, Academic Year 2024/2025. \\

% \section*{Team members and roles}


% copy-pasted from project proposal
%\section*{Motivation and project goal}

% Songs have the unique ability to engage people in ways that few other
% mediums can match. While beats and melodies contribute to this impact,
% it is often the lyrics that give songs their true emotional strength.
Lyrics serve as one of the main foundations of songs, playing a crucial role in expressing feelings in many different ways. The emotional tone of songs can serve various purposes, such as automatized playlist creation or songs' organization,
offering an alternative to the more traditional genre-based classification. \\
%Analyzing the emotional tone of song texts can give useful insights about individual mental states, cultural trends, social issues and more.
% The main goal of this project is to perform emotion detection on stanzas of songs.
The goal of this project is the development of 4 Machine Learning models that perform emotion detection on song lyrics stanzas. To obtain a deeper understanding of emotional fluctuations within the texts, the models assign emotion labels to individual stanzas instead of full songs.
The emotion labels are assigned based on Robert Plutchik's eight primary emotions (shown in figure~\ref{fig:primary_emotions}), offering a comprehensive range for representing diverse emotional states.\\
\begin{figure}[H]
    \centering
    \includegraphics[scale= 0.30]{pictures/plutchik_primary_emotions.png}
    \caption{Plutchik's eight primary emotions}
    \label{fig:primary_emotions}
\end{figure}

This report tries to cover and illustrate clearly various aspects of this work. 
In the \textit{Method} section, we will provide a detailed explanation of the data and procedures used in the project, in particular describing the pipeline taken to implement the models.
Then, the \textit{Results} chapter will provide an overview of the obtained results with the aid of plots and figures, highlighting the significant outcomes.
This section will be connected to the last two, i.e. the \textit{Discussion} and \textit{Conclusions} ones, which will explain what the general findings mean, recapping the primary objective of the work and discussing the importance or potential applications of the results.
% Furthermore, we intend to compare different textual preprocessing and
% Machine Learning models, in order to explore different possibilities and
% evaluate their performances.


%\section*{Dataset overview}



% Dataset overview
% !!! cut, content in intro
% % !!!
% cut at the moment; content moved to introduction

\chapter{Dataset overview}
\label{ch:capitolo1}

% copy-pasted from project proposal
The dataset utilized in this project represents a subset of songs
derived from the Genius Song Lyrics Dataset\textsuperscript{\cite{geniusdataset}}.
The dataset contains 11 attributes
that represent various song data, including the lyrics.
The original dataset includes songs in all languages: for our aim
we will be using the english ones only.\\

The dataset doesn't have emotion labels, which are essential for training the models.
To create the ground truth, the model
Albert Base v2\textsuperscript{\cite{albert-base-v2}} was used, classifying
stanzas' lyrics with Plutchik's eight primary emotions.

% The stanzas are labeled using Robert Plutchik's 8 primary emotions;
% the emotions included in this representation are:
% anger, fear, sadness, disgust, surprise, anticipation, trust, and joy.
% Such multifaceted emotions allow us to finely analyze the feelings and
% moods conveyed by songs.
% 
% \clearpage

% IN REALTA' DOVREBBE ESSERE LA SEZIONE "METHODS" 
\chapter*{Methods}
\label{ch:capitolo2}

% This section will provide an overview about the data, methods and procedures used in the project.\\

The dataset used in this project is a sampled subset of English-language
songs derived from the \textit{Genius Song Lyrics Dataset}\textsuperscript{\cite{geniusdataset}}.
% The original dataset (3m records) included songs in many different languages;
% however, this work focused exclusively on English-language ones.
The original dataset contained numerous attributes; the ones considered
relevant for model training are:
\begin{itemize}
    \item \textbf{title:} the song's title;
    % \item \textbf{artist:} the artist
    % \item \textbf{year:}
    % \item \textbf{views:}
    % \item \textbf{features:}

    % \item \textbf{id:}
    \item \textbf{lemmatized\_stanzas:} lyrics of the single stanza;
    
    \item \textbf{stanza\_number:} identifies the position of the stanza in the song;

    \item \textbf{is\_chorus:} boolean variable that attests whether the stanza is
        a chorus or not;
    
    \item \textbf{tag:} represents the genre of the song. For easier handling,
        this attribute of the original dataset has been one-hot encoded into various boolean variables
        (is\_country, is\_pop, is\_rap, is\_rb, is\_rock);

    \item \textbf{label:} represents the emotional classification of the stanza,
        assigned by Albert Base v2\textsuperscript{\cite{albert-base-v2}} model.
    
    % \item \textbf{is\_country:}
    % \item \textbf{is\_pop:}
    % \item \textbf{is\_rap:}
    % \item \textbf{is\_rb:}
    % \item \textbf{is\_rock:}
\end{itemize}

% The dataset originally didn't contain emotion labels, essential for training the models.
% To create the ground truth, the model
% Albert Base v2\textsuperscript{\cite{albert-base-v2}} was used, classifying
% stanzas' lyrics into Plutchik's eight primary emotions.
All of these attributes, except for \texttt{title}, were the result
of the preprocessing phase, as described in section~\ref{preprocessing}.
Due to limited computational power, the labeling process was time-intensive,
ultimately resulting in a limited dataset consisting of
\textbf{(QUANTE? AGGIUNGEREI NUMERO STROFE)}.


\section*{Preprocessing}
\label{preprocessing}
The initial preprocessing step involved sampling from the original dataset
while maintaining the proportional distribution of genres.
This approach ensured that the genre representation in the sampled subset
accurately reflected that of the full dataset.\\

% The preliminary text cleaning process focused on the \texttt{lyrics} attribute,
% which was the attribute of the original dataset that contained
% the entire lyrics of each song (in string format). Initially, we built a RegEx
% to clean the lyrics' strings from noise, specifically targeting words enclosed
% between square brackets that were irrelevant for the stanza splitting process.
% Many of the keywords marking different stanzas were written within square brackets,
% and removing the non-keyword items within brackets was essential to prevent
% potential issues. \\
% The crucial step was the stanza splitting. After cleaning the strings from the
% noisy square-bracketed items, we split them based on various keywords used to
% denote stanzas (such as \texttt{chorus, verse, intro, outro, refrain, hook} etc.). 
% The RegEx we developed also accounted for the different formats in which these
% keywords appeared; between square brackets, parentheses, without brackets, only a
% double newline character between one stanza and the other.
% The output of this step was, for each song record, a list of stings, corresponding
% to a list of stanzas (with the stanza's header as the corresponding keyword). \\
% Next, we removed the resulting strings that were uninformative; such as empty
% strings or those with fewer than 20 characters, which were too short to provide
% useful content. \\
% As a result, the output of this preliminary preprocessing phase is a dataset in
% which the records are not whole songs anymore but single stanzas; each numbered
% based on its position in the song. \\

The preliminary text cleaning process focused on the \texttt{lyrics} attribute,
which contained the complete lyrics of each song in string format.
Initially, a regular expression (RegEx) was built to remove noise from the
lyrics, specifically targeting words enclosed in square brackets that were
irrelevant to the stanza splitting process. Many keywords marking different
stanzas were written within square brackets, and removing non-keyword items
inside brackets was crucial to avoid potential issues.\\

The next critical step was stanza splitting. After cleaning texts from
noisy square-bracketed items, lyrics were split based on various keywords
used to denote stanzas (such as "chorus", "verse", "intro", "outro", "refrain", "hook", etc.).
The RegEx developed accounted for the different formats in which these keywords
appeared, including square brackets, parentheses, or no brackets at all, as well
as stanzas separated only by double newline characters.
The output of this step was, for each song record, a list of strings
representing individual stanzas (each stanza has also a header with the corresponding
keyword; this aspect will be discussed in the next paragraph).
Next, uninformative strings—such as empty strings or those with fewer
than 20 characters—were removed, as they were too short to provide meaningful
content.
As a result, the output of this preliminary preprocessing phase was a dataset
where the records were no longer whole songs but individual stanzas, each
numbered according to its position within the song.\\

% A further and deeper cleaning process on the stanzas involved the creation of the
% boolean feature \texttt{is\_chorus}; \texttt{true} value for repeated stanzas for
% the same song or stanzas that had \texttt{hook, chorus, refrain, bridge} as a
% header. \\
% We then removed the stanza headers and the newline characters between verses to
% obtain cleaner stanzas. \\ 
% Since choruses, hooks, bridges and refrains often repeat throughout songs, we
% decided to drop duplicate stanzas in order to avoid redundant data.
% This resulted in a dataset of cleaned and non-duplicate stanzas: the checkpoint
% for the labelling step and the starting point for the text lemmatization process.
% To label the dataset, the Albert Base v2 model has been used; this transformer
% model is primarily aimed at being fine-tuned on tasks that use the whole sentence
% to make decisions, such as sequence classification. 
% \\
% The next step involved lemmatizing stanzas using the \texttt{spaCy} library. We
% created a list of lemmatized tokens (filtering punctuation and empty words). 
% We opted for lemmatization over stemming because lemmatization produces more
% accurate and meaningful results, particularly for tasks requiring semantic
% understanding, such as in our case.


A further and more detailed cleaning process on the stanzas involved the creation
of the boolean feature \texttt{is\_chorus}, which was assigned a \texttt{true}
value for repeated stanzas within the same song or for stanzas with headers such
as "hook", "chorus", "refrain", or "bridge".
Next, stanza headers and newline characters between verses were removed to obtain
cleaner stanzas.
Since choruses, hooks, bridges, and refrains often repeat throughout songs,
duplicate stanzas were discarded to avoid redundant data. This resulted in a
dataset of cleaned, non-duplicate stanzas, which served as the checkpoint for
the labeling step and the starting point for the text lemmatization process.\\

The subsequent step involved lemmatizing the stanzas using the \texttt{spaCy}
library. A list of lemmatized tokens was created by filtering out punctuation
and empty words. Lemmatization was chosen over stemming because it produces
more accurate and meaningful results, particularly for tasks requiring semantic
understanding, such as the one at hand.\\

Since the dataset was not pre-labeled at the stanza level, ALBERT Base v2 was employed for this task.
This transformer model is specifically designed to be fine-tuned on tasks that
require an understanding of the entire sentence, such as sequence classification.\\


\section{Models developement}
\section{Evaluation}



\clearpage

% static models
% \chapter{Static Models}
\label{ch:static_models}


\section{Random Forest}

\section{SVM}


% \clearpage

% neural networks
% \chapter{Neural Networks}
\label{ch:capitolo4}

\section{One-Dimensional Convolutional Neural Network}

\section{Recurrent Neural Network}

% \clearpage

\chapter*{Conclusions}
\label{ch:conclusions}
% This study aimed at demonstrating how emotion detection in song
% lyrics stanzas can provide valuable insights into the emotional landscape of
% music and how it can be implemented with Machine Learning models. 
% These findings have practical applications, such as improving music
% recommendation systems and creating mood-based playlists.
% At the same time, the study faced challenges, particularly with interpreting
% ambiguous or context-dependent lyrics, which highlights opportunities for further
% research in this field.
This study aimed at exploring various Machine Learning techniques perform an
emotion detection task on songs, which are irregular, complex texts.
The particular field has many practical applications, such as improving
recommendation systems.

The results of the project point towards better performances obtained by more
straightforward, simpler models; neural networks generally struggled to find
general, meaningful patterns and correlations, leading into suboptimal
training and testing performances. These results might have been caused by a
series of factors, such as the likely feature overlap between different classes and
unreliable labeling.\\

As mentioned in the previous chapter, there are things that can be done to
solve both these issues, such as using alternative models, techniques or
sources for generating the ground truth. Another possible approach
is to use a probabilistic approach for the labeling process, which
can indeed guide models into more informed decisions.\\

% In conclusion,
% %since the static models performed better in this particular study,
% through the exploration of different techniques for emotion detection and
% the strengthening of the ground truth,
% it may be possible to achieve better performance on the emotion
% detection task, ultimately advancing its applicability in real-world scenarios.

In conclusion, emotion detection can improve by refining techniques and enhancing the quality of data.
Addressing key issues such as feature overlap and unreliable labels
could lead to more robust models, ultimately improving their
usefulness in real-world applications such as music recommendation systems
and content curation across various media platforms.
\clearpage

% \renewcommand{\contentsname}{Bibliography}
% \bibliographystyle{plain}
% \bibliography{bibliography}
% \addbibresource{bibliography.bib}
\printbibliography
\renewcommand{\listfigurename}{List of figures}
\listoffigures

% \thispagestyle{empty}
% \clearpage


\end{document}
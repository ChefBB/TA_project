\chapter*{Results}
\label{ch:results}
% The performances of the models were evaluated using the
% \texttt{classification\_report} function from the \texttt{scikit-learn} library. 
% This function is particularly useful as it offers an overview of key evaluation
% metrics commonly used in Machine Learning, i.e. accuracy, precision, recall,
% and F1-score.\\
The performance metrics taken into account for evaluation are the following:
accuracy, precision, recall, and F1-score. Taking all of them into account
is crucial to accurately evaluate how well each model performs.

For the static models implemented in this project, the classification report
revealed an accuracy of 34\% for the Random Forest algorithm and 43\% for SVM. 
These results can be considered reasonable, given that the task at hand is a
multi-class classification problem with 8 classes.\\

As mentioned in the previous chapter, the development of neural networks
iterated testing phases and adjustments over different 
Many different configurations for data splitting, preprocessing, architectures and
training parameters were tested.\\

Neural Networks' better configurations were obtained through experimentation with
various parameters, performances were not on par with the ones obtained by the
static models. 

\textbf{AGGIUNGERE ALTRO}
\chapter{Dataset overview}
\label{ch:capitolo1}

% copy-pasted from project proposal
The dataset used for the project is a set of lyrics of songs extracted
from the Genius Song Lyrics Dataset (\cite{geniusdataset}).
The dataset contains 11 attributes
that represent various song data, including the lyrics. It
has a total of over 3 million songs by 641,349 different artists.
The original dataset includes songs in all languages: for our aim
we will be using the english ones only. For now we are keeping
every other column including, of course, the lyrics one, since
we might be able to use them. Then, each song will be divided
into its composing stanzas, which are the actual subject of our analysis. 

In order to create the ground truth to train the models, we will be
using the already existing Meta's Bart large Multi Natural Language
Inference (MNLI). We will be using Robert Plutchik's 8 primary emotions
as labels; the emotions included in this representation are:
anger, fear, sadness, disgust, surprise, anticipation, trust, and joy.
Such multifaceted emotions allow us to finely analyze the feelings and
moods conveyed by songs.
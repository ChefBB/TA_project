\chapter*{Abstract}
% Analyzing the emotional tone of songs texts can give insights into
% societal trends and this information can be useful especially for recommendation
% algorithms.
% This study aims to build four Machine Learning models that classify
% emotions expressed in English song lyrics at the stanza level;
% the emotion labels are given according to
% Plutchik's eight primary emotions.\\

% The chosen module architectures are the following:
% \begin{itemize}
%     \item \textbf{Random Forest}
%     \item \textbf{SVM}
%     \item \textbf{One-Dimensional Convolutional Neural Network}
%     \item \textbf{Recurrent Neural Network}
% \end{itemize}

% These were chosen for their popularity and wide applicability, as well as
% using different conceptual approaches to solve the problem.
This study focuses on developing four Machine Learning models to classify
emotions conveyed in English song lyrics at the stanza level. The
classification leverages Plutchik's framework of eight primary emotions,
offering a nuanced understanding of emotional expression in lyrical content.

The selected model architectures are as follows:
\begin{itemize}
    \item \textbf{Random Forest}
    \item \textbf{Support Vector Machine (SVM)}
    \item \textbf{One-Dimensional Convolutional Neural Network (1D-CNN)}
    \item \textbf{Recurrent Neural Network (RNN)}
\end{itemize}

These models were chosen for their proven effectiveness across various
domains and their diverse approaches, providing a thorough
investigation of different techniques for emotion classification in text.\\

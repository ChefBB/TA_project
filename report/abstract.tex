\firstchapter
\chapter*{Abstract}
\label{ch:abstract}
Lyrics serve as one of the main foundations of songs, playing a crucial role in expressing feelings in many different ways.
The emotional tone of songs can also serve various purposes, such as automatized playlist creation or songs' organization,
offering an alternative to the more traditional genre-based classification. This study focuses on developing four Machine Learning models
to classify emotions conveyed in English song lyrics, at the stanza level.
The models were chosen for their proven effectiveness across various domains and
their diverse approaches, providing a thorough investigation of different
techniques and depths for emotion classification in text.\\
The study begins with the preprocessing of the \textit{Genius Song Lyrics}\textsuperscript{\cite{geniusdataset}} dataset.
% This was a deep and time-consuming process, which ultimately resulted in a complete reshape of the originale dataset. 
The models were then trained through transfer learning, by generating the ground
truth using the already existing ALBERT Base v2 model.
% The ground truth was created with an already existing model: Albert Base v2, which was chosen to avoid the computational costs of larger models like BERT. The resulting labelled dataframe was then splitted following the train/test split protocol.
% The study proceeded with the training and the performance analysis of the following models: Random Forest and Support Vector Machine, followed by a One-Dimensional Convolutional Neural Network and a Recurrent Neural Network. 
The results are discussed to highlight the specific challenges encountered.
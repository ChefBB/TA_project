\chapter*{Conclusions}
\label{ch:conclusions}
% This study aimed at demonstrating how emotion detection in song
% lyrics stanzas can provide valuable insights into the emotional landscape of
% music and how it can be implemented with Machine Learning models. 
% These findings have practical applications, such as improving music
% recommendation systems and creating mood-based playlists.
% At the same time, the study faced challenges, particularly with interpreting
% ambiguous or context-dependent lyrics, which highlights opportunities for further
% research in this field.
This study aimed at exploring various Machine Learning techniques perform an
emotion detection task on songs, which are irregular, complex texts.
The particular field has many practical applications, such as improving
recommendation systems.

The results of the project point towards better performances obtained by more
straightforward, simpler models; neural networks generally struggled to find
general, meaningful patterns and correlations, leading into suboptimal
training and testing performances. These results might have been caused by a
series of factors, such as the likely feature overlap between different classes and
unreliable labeling.\\

As mentioned in the previous chapter, there are things that can be done to
solve both these issues, such as using alternative models, techniques or
sources for generating the ground truth. Another possible approach
is to use a probabilistic approach for the labeling process, which
can indeed guide models into more informed decisions.\\

% In conclusion,
% %since the static models performed better in this particular study,
% through the exploration of different techniques for emotion detection and
% the strengthening of the ground truth,
% it may be possible to achieve better performance on the emotion
% detection task, ultimately advancing its applicability in real-world scenarios.

In conclusion, emotion detection can improve by refining techniques and enhancing the quality of data.
Addressing key issues such as feature overlap and unreliable labels
could lead to more robust models, ultimately improving their
usefulness in real-world applications such as music recommendation systems
and content curation across various media platforms.